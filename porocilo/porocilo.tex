\documentclass[11pt,a4paper]{article}

\usepackage[slovene]{babel}
\usepackage[utf8x]{inputenc}
\usepackage{graphicx}
\usepackage{hyperref}

\pagestyle{plain}

\begin{document}
\title{Poročilo pri predmetu \\
Analiza podatkov s programom R}
\author{Rok Zadravec}
\maketitle

\section{ZDA - populacija}

Tema mojega projekta je analiza vseh posameznih držav iz ZDA. V projektu bom za vsako državo podal: njeno glavno mesto - imenska, največje mesto - imenska, njeno površino (podano v $km^2$) - številska, njeno površino zemlje (podano v $km^2$) - številska, njeno populacijo -številska, številska delež prebivalstva ZDA številska, in ali po velikosti sodi med večje ali manjše- urejenostna.

Moj cilj je pridobiti idejo o porazdelitvi prebivalstva v celotni ZDA prav tako kot gostoti.

Podatke sem pridobil na strani Wikipedie \url{http://en.wikipedia.org/wiki/List_of_states_and_territories_of_the_United_States} in \url{http://en.wikipedia.org/wiki/List_of_U.S._states_and_territories_by_population}.

\section{Obdelava, uvoz in čiščenje podatkov}

Podatke sem uvozil v obliki csv (podatki ločeni z vejico). Te podatke sem prej imel shranjene v Excel-ovi tabeli. Podatke sem
pridobil iz strani \url{http://en.wikipedia.org/wiki/List_of_U.S._states_and_territories_by_area} ter
\url{http://en.wikipedia.org/wiki/List_of_states_and_territories_of_the_United_States}. Nekatere podatke sem odstranil, in vse skupaj podal v eni sami tabeli. Tabela je shranjena v mapi "Podatki".
Ko poženemo program v mapi uvoz.r ta vrne urejeno tabelo.

\section{Analiza in vizualizacija podatkov}

Grafe poženemo z programom grafi.R ki je v mapi Slike. Pdf slika grafa se shrani v mapo
grafi.pdf,  ki je prav tako v mapi Slike.

\includegraphics{../slike/grafi.pdf}

\section{Napredna analiza podatkov}

%\includegraphics{../slike/naselja.pdf}

\end{document}
