\documentclass[11pt,a4paper]{article}

\usepackage[slovene]{babel}
\usepackage[utf8x]{inputenc}
\usepackage{graphicx}

\pagestyle{plain}

\begin{document}
\title{Poročilo pri predmetu \\
Analiza podatkov s programom R}
\author{Rok Zadravec}
\maketitle

\section{Izbira teme}

Tema mojega projekta je analiza vseh posameznih držav iz ZDA. V projektu bom za vsako državo podal: njeno glavno mesto - imenska, največje mesto - imenska, njeno površino (podano v mi^2 in km^2) - številska, njeno površino zemlje (podano b mi^2 in km^2) - številska, njeno populacijo -številska, številska delež prebivalstva ZDA  - številska, in ali po velikosti sodi med večje ali manjše - urejenostna.

Moj cilj je pridobiti idejo o porazdelitvi prebivalstva v celotni ZDA prav tako kot gostoti.

Podatke sem pridobil na strani Wikipedie http://en.wikipedia.org/wiki/List_of_states_and_territories_of_the_United_States in 
http://en.wikipedia.org/wiki/List_of_U.S._states_and_territories_by_population.


\section{Obdelava, uvoz in čiščenje podatkov}

Podatke sem uvozil v obliki csv. Te podatke sem prej imel shranjene v Excel tabeli. Podatke sem pridobil iz strani http://en.wikipedia.org/wiki/List_of_U.S._states_and_territories_by_area ter http://en.wikipedia.org/wiki/List_of_states_and_territories_of_the_United_States. Nekatere podatke sem odstranil, in vse skupaj podal v eni sami tabeli.

\section{Analiza in vizualizacija podatkov}

\includegraphics{../slike/povprecna_druzina.pdf}

\section{Napredna analiza podatkov}

\includegraphics{../slike/naselja.pdf}

\end{document}
